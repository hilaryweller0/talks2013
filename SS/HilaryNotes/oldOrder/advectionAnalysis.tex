\chapter{Numerical Analysis of Advection Schemes}
\label{chap:advectionAnalysis}

\section{Von-Neumann Stability Analysis}

Von Neumann stability analysis gives us an equation for the amplification factor, $A$, of a numerical scheme for solving an equation which tells us the following about the scheme:\\
{\renewcommand{\arraystretch}{1.1}\begin{tabular}{ll}
$|A|<1$ & stable and damping\\
$|A|=1$ & neutrally stable \\
$|A|>1$ & unstable (amplifying)\\
\end{tabular}

The scheme is:

\begin{tabular}{ll}
conditionally stable & if stable only for sufficiently small time step\\
unconditionally stable & if stable for all time steps\\
unconditionally unstable & if unstable for all time steps\\
\end{tabular}

In general, $A$ depends on the time step and on the resolution of the  wavenumbers of the solution. We know that the solution, $\phi$, can be represented as an infinite Fourier series
\begin{equation}
\phi = \sum_{k=-\infty}^{\infty} A_k e^{ikx}.
\end{equation}
where $A_k$ are the \opttext{Fourier coefficients} and $k$ are the \opttext{wavenumbers}.

Linear advection will advect each of the Fourier modes independently and any {\it linear} scheme for linear advection such as FTBS or CTCS will also advect each of the Fourier modes independently. So we should find how a scheme treats one mode, $e^{ikx}$. So assume that $\phi_j^{(n)}=e^{ikj\Delta x}$ (since at position $j$, $x=j\Delta x$) and find the amplification factor $A$ such that:
\begin{equation}
\phi_j^{(n+1)} = A \phi_j^{(n)}
\end{equation}


\section{Von-Neumann Stability Analysis of FTBS}

FTBS for linear advection is:
\begin{equation}
\phi^{(n+1)}_j = \phi^{(n)}_j -c\bigl(\phi^{(n)}_j - \phi^{(n)}_{j-1}\bigr)
\end{equation}
Substitute in $\phi_j^{(n)}=e^{ikj\Delta x}$ and $\phi_j^{(n+1)} = A \phi_j^{(n)}$:

\begin{equation}\opttext{
Ae^{ikj\Delta x} = e^{ikj\Delta x} - c(e^{ikj\Delta x} - e^{ik(j-1)\Delta x})}
\end{equation}
Cancel powers of $e^{ikj\Delta x}$ and rearrange to find $A$ in terms of $c$ and $k\Delta x$:
\begin{equation}\opttext{
A = 1 - c(1 - e^{-ik\Delta x})}
\end{equation}
We need to find the magnitude of $A$ so we need to write it down in real and imaginary form. So substitute $e^{-ik\Delta x} = \cos k\Delta x - i\sin k\Delta x$:
\begin{equation}\opttext{
A = 1 - c(1-\cos k\Delta x) - ic\sin k\Delta x}
\end{equation}
and calculate $|A|^2=AA^*$ ($A$ multiplied by its complex conjugate):


For what values of $\Delta t$ is $|A|\le 1$? \\ \ \\
Remember that $A$ is complex and multiply by its complex conjugate, $A^*$:
%\vspace{0.5cm}
\begin{align*}
&|A|^2 = 1 - 2c(1-\cos k\Delta x) + c^2(1-2\cos k\Delta x + \cos^2 k\Delta x) + c^2 \sin^2 k\Delta x\\ %&\\
&\implies |A|^2 = 1 -2c(1-c)(1-\cos k\Delta x)
\end{align*}
We need to find for what value of $c$ is $|A|\le 1$ in order to find when FTBS is stable. 

\clearpage
\begin{minipage}{0.58\linewidth} \raggedright
For FTBS, \\
$|A|^2 = 1 -2c(1-c)(1-\cos k\Delta x)$

FTBS will be most unstable when $|A|$ is maximised. This occurs when $\cos k\Delta x=-1$. Ie when $k\Delta x=\pm\pi$.\\
So FTBS is stable when:\\
\begin{center}
$1 -4c(1-c)\le 1$
\end{center}

\hfill
$|A|^2$ for FTBS for $k\Delta x=\pi$ $\rightarrow$
\end{minipage}
\begin{minipage}{0.4\linewidth}
\includegraphics[width=\linewidth]{pythonWork/stabAnal/FTBSA.pdf}
\end{minipage}

So FTBS is unstable if $u<0$ or if $\frac{u\Delta t}{\Delta x}>1$. We will now define:\\
\begin{tabular}{lll}
\bf Upstream scheme & FTBS & when $u\ge 0$\\
                & FTFS & when $u< 0$ \\
\end{tabular}

The upstream scheme is first order accurate in space and time, conditionally stable, damping and artificially diffusive.

\subsection{What should $|A|$ be for real linear advection?}
\label{secn:AforRealAdvection}

For initial conditions consisting of a single Fourier mode:
\begin{equation*}
\phi(x,0) = A_k e^{ikx}
\end{equation*}
The solution at time $t$ is:
\begin{equation*}
\phi(x,t) = \opttext{A_k e^{ik(x-ut)}.}
\end{equation*}
This can be represented as a factor times $e^{ikx}$:
\begin{equation*}
\phi(x,t) = A_k e^{-ikut} e^{ikx}
\end{equation*}
So if $t=n\Delta t$ we have $A^n = e^{-ikut} = e^{-ikun\Delta t}$. Therefore $A=e^{-iku\Delta t}$ which has $|A|=1$.
}

\clearpage
\section{Von Neumann Stability Analysis of CTCS}

The amplification factor, $A$, can also be found by substituting $\phi^{(n)}_j = A^n e^{ikj\Delta x}$ into the equation for the numerical scheme. So for CTCS:
\begin{equation}
\phi^{(n+1)}_j = 
\phi_j^{(n-1)} - c\bigl(\phi^{(n)}_{j-1} - \phi^{(n)}_{j-1}\bigr)
\end{equation}
we get:
\begin{align*}
&\opttext{A^{n+1}e^{ikj\Delta x} - A^{n-1}e^{ikj\Delta x} + c(A^{n}e^{ik(j+1)\Delta x} - A^{n}e^{ik(j-1)\Delta x}) = 0} \\
\opttext{\implies}&\opttext{A^2 + (2ic\sin k\Delta x) A - 1 = 0}\\
\implies&A=\opttext{-ic\sin k\Delta x \pm \sqrt{1-c^2 \sin^2 k\Delta x}}
\end{align*}

No point in considering $k\Delta x=\pm\pi$ because trivially $A=\pm 1$. Instead consider $k\Delta x=\pi/2$:

\begin{minipage}{0.4\linewidth}
\includegraphics[width=\linewidth]{pythonWork/stabAnal/CTCSA.pdf}
\end{minipage}
\begin{minipage}{0.58\linewidth}
\begin{itemize}\raggedright
\item There are two roots, a physical mode and a spurious computational mode
\item There is a computational mode because of the three time-levels
\item CTCS is neutrally stable for $|c|\le 1$. This is really good!
\item CTCS is unstable for $|c|>1$.
\item The neutral stability means that waves of any wavelength do not amplify or decay. So CTCS is not diffusive.
\item However short waves propagate too slowly. This leads to dispersion errors
\end{itemize}
\end{minipage}

