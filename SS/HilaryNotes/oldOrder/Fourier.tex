\chapter{Fourier Series}

\section{Definitions}

\begin{minipage}{0.65\linewidth}
Any periodic, integrable function, $f(x)$ (defined on $[-\pi,\pi]$), can be expressed as an inifinite sum of sines and cosines:
\begin{equation}
f(x) = \sum_{n=0}^{\infty} a_n \cos nx + \sum_{n=0}^{\infty}b_n \sin nx
\label{eqn:FourierSinCos}
\end{equation}
\end{minipage}
%
\begin{figure}[ht!]
\subfloat[The first three Fourier modes of a noisy function.]
{\label{fig:smoothFunc}
\includegraphics[width=0.45\linewidth]{figs/smoothFunc.png}}
\vfill \vspace{-3.2in}\hfill
\subfloat[The first four Fourier modes of a square wave.  The additional oscillations are {\em``spectral ringing''}]
{\label{fig:squareWave}.
\includegraphics[width=0.34\linewidth]{figs/squareWave-Fourier_Series.png}}
\caption{Truncated Fourier decompositions of two functions}
\label{fig:Fourier}
\end{figure}
\\
The more Fourier modes that included, the closer their sum will get to the original function.

\clearpage

Equivalently, equation \ref{eqn:FourierSinCos} can be expressed as an infinite sum of exponentials:
\begin{equation}
f(x) = \sum_{n=0}^{\infty} a_n \cos nx + \sum_{n=0}^{\infty}b_n \sin nx
     = \sum_{n=-\infty}^{\infty} A_n e^{inx}.
\label{eqn:FourierExp}
\end{equation}
The $a_n$, $b_n$ and $A_n$ are the Fourier coefficients and the sines and cosines are the Fourier modes.

The Fourier coefficients can be calculated using:
\begin{equation*}
a_n = \frac{1}{\pi}\int_{-\pi}^{\pi}f(x)\cos(nx)dx ~, ~~
b_n = \frac{1}{\pi}\int_{-\pi}^{\pi}f(x)\sin(nx)dx
\end{equation*}

On a computer this is done with a {\bf Fast Fourier Transform} (or {\tt fft}):\\
Given the value of a function, $f$, at $N$ locations the Fourier transform calculates the first $N$ Fourier coefficients. This is the transformation to spectral space. The inverse Fourier transform (sometimes called {\tt ifft}) transforms the Fourier coefficients back to the $f$ values (transforming from spectral back to real space):

\begin{tabular}{ccc}
$f_1, f_2, \cdots f_N$ & $\xrightarrow{\text{\tt fft}}$ & $A_0, A_1, \cdots A_N$ \\
$A_0, A_1, \cdots A_N$ & $\xrightarrow{\text{\tt ifft}}$ & $f_1, f_2, \cdots f_N$ \\
\end{tabular}

\clearpage
\section{Differentiation and Interpolation}

If we know the Fourier coefficients, $A_n$, of a function $f$ then we can calculate the gradient of $f$ at any point, $x$: If
\begin{equation}
f(x) = \sum_{n=0}^{\infty} a_n \cos nx + \sum_{n=0}^{\infty}b_n \sin nx
= \sum_{n=-\infty}^{\infty} A_n e^{inx}
\end{equation}
then
\begin{equation}
f^\prime(x) = 
\opttext{
\sum_{n=0}^{\infty} -n a_n \sin nx + \sum_{n=0}^{\infty}n b_n \cos nx
= \sum_{n=-\infty}^{\infty} i~n~A_n e^{inx}.}
\label{eqn:FourierGrad}
\end{equation}
and the second derivative:
\begin{equation}
f^{\prime\prime}(x) = \opttext
{
\sum_{n=0}^{\infty} -n^2 a_n \cos nx - \sum_{n=0}^{\infty}n^2 b_n \sin nx
\sum_{n=-\infty}^{\infty} -n^2~A_n e^{inx}.
}
\label{eqn:FourierGrad2}
\end{equation}
These have spectral accuracy; the order of accuracy is as high as the number of points. Similarly equation \ref{eqn:FourierSinCos} or  \ref{eqn:FourierExp} can be used directly to interpolate $f$ onto an undefined point, $x$. Again, the order of accuracy is spectral. 

\clearpage
\section{Spectral Models}
\begin{itemize}
\item ECMWF use a spectral model. 
\item The prognostic variables are transformed between physical and spectral space using {\tt fft}s and {\tt ifft}s.
\item Gradients are calculated very accurately in spectral space
\end{itemize}

\section{Wave Power and Frequency}

\begin{minipage}{0.55\linewidth}
\setlength{\parskip}{6pt}
\setlength{\parindent}{0pt}
$\sin nx$ and $\cos nx$ are waves:

\setlength{\tabcolsep}{4pt}
\begin{tabular}{lcl}
wavenumber & $n$ & Number of complete waves \\
(or frequency) && that will fit into the interval \\
               &&$[-\pi,\pi]$\\
wavelength && $\lambda = \frac{2\pi}{n}$
\end{tabular}
\end{minipage}
\begin{minipage}{0.43\linewidth}
%\vspace{-0.4in}
\includegraphics[width=\linewidth]{figs/sineWaves.png}
\end{minipage}

\begin{itemize}
\item If a function, $f$ has Fourier coefficients, $a_n$ and $b_n$ then wavenumber $n$ has power $a_n^2+b_n^2$.

\item A plot of wave frequency versus power is referred to as the power spectrum.
\end{itemize}

\clearpage
\begin{minipage}{0.7\linewidth}\centering\bf
Daily rainfall at a station in the Middle East for 21 years\\
\includegraphics[width=\linewidth]{figs/rainSmoothed.pdf}
\end{minipage}
\begin{minipage}{0.29\linewidth}
Truncated Fourier filtered rainfall:
\begin{itemize}
\item very smooth (only low wavenumbers included)
\item includes negative values -- {\em``spectral ringing''}
\end{itemize}
\end{minipage}
%
\begin{center}\bf
Power Spectrum of Middle East Rainfall

\includegraphics[width=0.8\linewidth]
{figs/rainPower.pdf}
\end{center}
%
number per year = $n\times$365/total number of days

\clearpage
\heading{Time Series of the Quasi-biennial Oscillation (QBO)}

The Quasi-biennial oscillation (QBO) is an oscillation of the equatorial zonal wind between easterlies and westerlies in the tropical stratosphere which has a mean period of 28 to 29 months:

\includegraphics[width=\linewidth]{figs/qbo.pdf}

The power at 2 years can be seen in the Fourier power spectrum (peak at 0.5 per year):

\includegraphics[width=0.6\linewidth]{figs/qboPower.pdf}

