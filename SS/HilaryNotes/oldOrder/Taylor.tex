\chapter{Taylor Series and Numerical Approximations}

\section{Taylor Series}

A Taylor series gives an approximation of a function, $f$, at a position $x+\Delta x$, close to a known point, $x$, in terms of derivatives at $x$:
\begin{equation}
f(x+\Delta x) = f(x) + \Delta x f^\prime(x) + \frac{\Delta x^2}{2!} f^{\prime\prime}(x) + \frac{\Delta x^3}{3!} f^{\prime\prime\prime}(x) + \cdots
+ \frac{\Delta x^j}{j!} f^{(j)}(x) + \cdots
\label{eqn:Taylor}
\end{equation}
where $f^\prime(x) = \frac{df}{dx}(x)$, $f^{\prime\prime}(x) = \frac{d^2f}{dx^2}(x)$. If infinitely many terms are used then this approximation is exact. However if all terms of order $n$ are discarded then the error is approximately proportional to $\Delta x^n$ (assuming that $\Delta x$ is small) and the approximation is said to be $n^{th}$ order accurate. For example if we do not evaluate any of the terms in $\Delta x^3$ and higher then the approximation and error are:
\begin{equation}
f(x+\Delta x) \approx f(x) + \Delta x f^\prime(x) + \frac{\Delta x^2}{2!} f^{\prime\prime}(x) + O(\Delta x^3)
\label{eqn:Taylor3}
\end{equation}
The $+ O(\Delta x^3)$ terms means ``of order 3'' and means that the error, $\varepsilon$, is proportional to $\Delta x^3$, which becomes small very rapidly as $\Delta x$ decreases. 

\subsection{Exercise}

\begin{enumerate}
\item Write the third order approximation for $f(x-\Delta x)$ in terms of $f(x)$, $f^\prime(x)$ and $f^{\prime\prime}(x)$.
\optparagraph{
\begin{equation}
f(x-\Delta x) \approx f(x) - \Delta x f^\prime(x) + \frac{\Delta x^2}{2!} f^{\prime\prime}(x) + O(\Delta x^3)
\end{equation}}

\item Write the third order approximation for $f(x+\Delta x)$ in terms of $f(x-\Delta x)$, $f^\prime(x-\Delta x)$ and $f^{\prime\prime}(x-\Delta x)$.
\optparagraph{
\begin{equation}
f(x+\Delta x) \approx f(x-\Delta x) + 2\Delta x f^\prime(x-\Delta x) + 2\Delta x^2 f^{\prime\prime}(x-\Delta x) + O(\Delta x^3)
\end{equation}}
\end{enumerate}

\clearpage
\section{Numerical Differentiation}
\label{Taylor:diffIntro}

Consider a set of points, $x_0, x_1, \cdots x_j,\cdots x_n$ where $x_j = j\Delta x$ (the points are distance $\Delta x$ apart). Assume that we know the value of the function $f(x)$ at these points, as shown in figure \ref{fig:fx}.
%\afterpage{\clearpage
\begin{figure}[H]\begin{center}
\includegraphics[scale=0.45]{figs/fx.pdf}
\caption{Values of a function $f$ at points $x_0,~x_1,\cdots,x_j,\cdots$.} 
\label{fig:fx}
\end{center}\end{figure}%}
Some possible estimate of $f^\prime_j=f^\prime(x_j)$ are:\\
\centerline{\begin{tabular}{ccc}
forward difference & backward difference & centred difference \\
\opttext{$f^{\prime}_j \approx \frac{f_{j+1} - f_{j}}{\Delta x}$}
&
\opttext{$f^{\prime}_j \approx \frac{f_{j} - f_{j-1}}{\Delta x}$}
&
\opttext{$f^{\prime}_j \approx \frac{f_{j+1} - f_{j-1}}{2\Delta x}$}
\end{tabular}}

Taylor series can be used to derive estimates of derivatives and to find their order of accuracy.

\section{Taylor Series to find Finite Difference Gradients}
\label{Taylor:diff}

In order to use a Taylor series (below) to find an approximation for $f^\prime$
\begin{equation}
f(x+\Delta x) = f(x) + \Delta x f^\prime(x) + \frac{\Delta x^2}{2!} f^{\prime\prime}(x) + \frac{\Delta x^3}{3!} f^{\prime\prime\prime}(x) + \cdots
+ \frac{\Delta x^j}{j!} f^{(j)}(x) + \cdots
\label{eqn:Taylor2}
\end{equation}%\vspace{-5pt}
\begin{enumerate}
\item write down the knowns
\item consider {\em where} we want to find $f^\prime$
\item consider what order of accuracy we want
\item write down Taylor series for some of the knowns
\item eliminate the additional unknowns to find $f^\prime$
\end{enumerate}

\subsection{Example}

\begin{enumerate}
\item Assume that we know \begin{minipage}[t]{0.35\linewidth}
$f_j=f(x_j)$\\
$f_{j-1}=f(x_{j-1})=f(x-\Delta x)$\\
$f_{j+1}=f(x_{j+1})=f(x+\Delta x)$
\end{minipage}
\item and we want to find $f^\prime_j$.
\item For 3 knowns we wonder if we can get second order accuracy
\item We do not want to generate too many unknowns. We don't know $f^\prime_{j-1}$ or $f^\prime_{j+1}$ so no Taylor series about $x_{j-1}$ or $x_{j+1}$. So let's try Taylor series for $f_{j+1}$ and $f_{j-1}$ about $x_j$
\[
f_{j+1} = \opttext{f_j + \Delta x f^\prime_j + \frac{\Delta x^2}{2!} f^{\prime\prime}_j + \frac{\Delta x^3}{3!}f^{\prime\prime\prime}} \hspace{4cm} + O(\Delta x^4)
\]
\[
f_{j-1} = \opttext{f_j - \Delta x f^\prime_j + \frac{\Delta x^2}{2!} f^{\prime\prime}_j - \frac{\Delta x^3}{3!}f^{\prime\prime\prime} } \hspace{4cm} +O(\Delta x^4)
\]
\item Eliminate $f^{\prime\prime}_j$ by taking the difference of the two equations\\
\[
f_{j+1} - f_{j-1} = \opttext{2\Delta x f^\prime_j + \frac{\Delta x^3}{3}f^{\prime\prime\prime}_j} \hspace{4cm} + O(\Delta x^4)
\]
Rearrange to get $f^\prime_j$
\[
f^\prime_j = \opttext{\frac{f_{j+1} - f_{j-1}}{2\Delta x} + \frac{\Delta x^2}{3!}f^{\prime\prime\prime}_j }+ O(\Delta x^3)
\]
We cannot elliminate $f^{\prime\prime\prime}_j$ so this is part of the error:
\[
f^\prime_j = \frac{f_{j+1} - f_{j-1}}{2\Delta x} + O(\Delta x^2)
\]
The error, $\varepsilon$, is proportional to $\Delta x^2$ ($\varepsilon \propto \Delta x^2$) so this approximation is second order accurate.
\end{enumerate}

\subsection{Exercise}

Use the Taylor series to find an approximation for $f^\prime_j$ in terms of $f_j$ and $f_{j-1}$. What order accuracy is it?

\opttext{
$f_{j-1} = f_j - \Delta x f_j^\prime + \frac{\Delta x^2}{2!} f_j^{\prime\prime}
+ O(\Delta x^3)
$
\\
$\implies f_j^\prime = \frac{f_j - f_{j-1}}{\Delta x} + \frac{\Delta x}{2!} f_j^{\prime\prime} + O(\Delta x^2)$
}


\clearpage
\section{Interpolation}
\label{secn:interp}

Interpolation of values from known points onto unknown points is an essential tool for numerical modelling. If the value of $f$ is known at points $x_1$ and $x_2$ (values $f_1$ and $f_2$) and we want to estimate the value of $f$ at point $x_i$ in between $x_1$ and $x_2$ then we can use linear interpolation (fig. \ref{fig:linInterp} (ie we assume that $f_i$ lies on a straight line between $f_1$ and $f_2$):
\begin{figure}[H]\raggedright
\begin{minipage}{0.49\linewidth}
\includegraphics[width=\linewidth]{figs/linInterp.pdf}
\end{minipage}
\hspace{0.05\linewidth}
\begin{minipage}{0.44\linewidth}\raggedright
Gradient $=$ \opttext{$\frac{f_2-f_1}{x_2-x_1}$} \\
$\implies f_i = f_1 + $ \opttext{$\beta\Delta x \frac{f_2-f_1}{x_2-x_1}$} \\
where $\Delta x = x_2-x_1$, $\beta=\opttext{\frac{x_i-x_1}{x_2-x_1}}$\\ \ \\
$\implies f_i = (1-\beta)f_1 + \beta f_2$\\
\end{minipage}
\caption{Linear interpolation of $f$ between $x_1$ and $x_2$.} 
\label{fig:linInterp}
\end{figure}
\begin{itemize}
\item If $f$ is known at $n$ points then a polynomial of degree $n-1$ can be fit to estimate $f$ at another point
\end{itemize}
Eg. Cubic Lagrange interpolation for constant grid spacing, $\Delta x$:
\begin{align}
\hat{f}(x)= -& \frac{1}{6}\beta(1-\beta)(2-\beta)f_{k-1}
             + \frac{1}{2}(1+\beta)(1-\beta)(2-\beta)f_k \nonumber\\
            +& \frac{1}{2}(1+\beta)\beta(2-\beta)f_{k+1}
             - \frac{1}{6}(1+\beta)\beta(1-\beta)f_{k+2}
\label{eqn:LagrangeInterp3}
\end{align}

