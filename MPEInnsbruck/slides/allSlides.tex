\makeatletter{}\newcommand{\HOME}{/home/hilary}
\newcommand{\FOAM}{\HOME/OpenFOAM/hilary-dev}
\newcommand{\RUN}{\FOAM/run}
\newcommand{\RUNWM}{\RUN/shallowWaterSphere}
\newcommand{\GMTC}{\HOME/gmt/usr/colours/special}


\newcommand{\de}{\ensuremath{^\circ}}
\newcommand{\prim}{\ensuremath{^\prime}}
\newcommand{\ie}{\ensuremath{i.e.}}

\newcommand{\sideLabel}[2]
{
\begin{sideways}\begin{minipage}{#1} \centering
    #2
\end{minipage}\end{sideways}
}

\renewcommand{\floatpagefraction}{0.7}
\renewcommand{\textfraction}{0}
\renewcommand{\topfraction}{1}
\renewcommand{\bottomfraction}{1}

\newcommand{\eqn}[1]{\mbox{$#1$}}
 
\makeatletter{}\makeatletter

\documentclass[mathserif,serif,handout]{beamer}
    

\usepackage{color, rotating, times, amssymb, amsmath, mathtools, graphicx, subfigure, tabularx, lscape, multirow, natbib, wasysym, array, hyperref,fancybox,animate}


\setlength{\tabcolsep}{3pt}
\renewcommand{\arraystretch}{1}
\newcolumntype{C}{>{\centering\arraybackslash}X}
\newcolumntype{K}{>{\centering\arraybackslash}p}
\newcolumntype{Y}{>{\raggedright\arraybackslash}X}

\setlength{\fboxsep}{1mm}


\definecolor{hsblue}{rgb}{0.2,0.2,0.6}
\definecolor{purple}{rgb}{0.6,0,0.7}
\definecolor{gray}{rgb}{0.7,0.7,0.7}
\definecolor{darkgreen}{rgb}{0.2,0.7,0.2}

\newcommand{\macroPath}{\HOME/latex/begin/foam}

\usepackage{latexsym,amssymb}
\usepackage{\macroPath/tensorCommon}        \usepackage{\macroPath/tensorEquation}      \usepackage{\macroPath/tensorOperator}      
\newcommand{\tims}{\negthinspace \times \negthinspace}
\newcommand{\eq}{\negthinspace = \negthinspace}
\newcommand{\plus}{\negthinspace + \negthinspace}
\newcommand{\minus}{\negthinspace - \negthinspace}
\newcommand{\lt}{\negthinspace <\negthinspace}
\newcommand{\gt}{\negthinspace > \negthinspace}
\newcommand{\dofs}{d.o.f.s}

\newcommand{\half}{{\ensuremath{^1\negthickspace/\negthinspace_2}}}
\newcommand{\hfrac}[2]{{\ensuremath{^{#1}\negthickspace/\negthinspace_{#2}}}}
\newcommand{\vortDiv}{\ensuremath{\sqrt{\zeta^2\plus \delta^2}}}
\newcommand{\expForm}[2]{\ensuremath{#1\tims 10^{#2}}}
\newcommand{\persec}{\ensuremath{s^{\minus 1}}}
\newcommand{\persecm}{\ensuremath{s^{\minus 1}m^{\minus 1}}}
\renewcommand{\dprod} {\cdot}
\everymath{\displaystyle}

\usepackage{pifont}
\newcommand{\tick}{\ding{52}}
\usepackage{wasysym}

\setlength{\leftmargini}{0em}
\setlength{\leftmarginii}{1em}



 



\makeatletter{}\title{Atmospheric Modelling on Arbitrary Grids}

\def\coauthors{Sean Cavany, Phil Browne, John Thuburn, Colin Cotter}

\author[me]{Hilary Weller\inst{1} \\ \vskip 1em
\footnotesize With thanks to \coauthors}

\institute{\inst{1} University of Reading}

\titlegraphic{\raggedright\vfill\vspace{2cm}
\includegraphics[scale=0.05]{\HOME/graphics/uniLogoName.jpg}\\
\includegraphics[scale=0.3]{\HOME/graphics/NCAS_logo.png}
}

\begin{document}
{
\usebackgroundtemplate{
    \begin{picture}(0,0)\put(220,-260)
    {\includegraphics[width=0.45\textwidth]{mesh_face.png}}\end{picture}}
\frame{\titlepage}
}
 

\makeatletter{}\begin{frame}
\begin{block}{Why?}
\pause
\begin{itemize}
\item Resolution where it is needed rather than at the poles
\item No abrupt changes in mesh/grid spacing
\item In order to move away from a lat-lon grid, the discretisation and code structure must inevitably be more complicated anyway   
\end{itemize}
\pause
\end{block}

\begin{block}{Why not?}
\pause
\begin{itemize}
\item Lat-lon grid model is simple and $\therefore$ efficient
\item Simple to achieve high accuracy
\item The most accurate forecast models (ECMWF and Met Office) both use lat-lon grids
\end{itemize}
\pause

\end{block}
\begin{block}{But}
\pause

\begin{itemize}
\item Lat-lon grid models do not scale to $\sim\negthinspace 100K$ processing cores
\item Errors associated with local area forecasts
\end{itemize}
\end{block}
\end{frame}

 
\makeatletter{}\begin{frame}
\frametitle{Parallel scaling bottlenecks}

\begin{minipage}{0.6\linewidth}
\includegraphics[width=\linewidth]
{../../../2011/pLinks/graphics/meshes+latLon+constant+mesh.pdf}

\begin{picture}(0,0)\put(79,76)
{\includegraphics[scale=0.42]{../../../2011/gungHoNCASJul2011/figs/arrow.pdf}}
\end{picture}

\end{minipage}
\hfill
\begin{minipage}{0.35\linewidth}

\begin{itemize}
\item lat-lon grid?
\item semi-implicit?
\item semi-Lagrangian?
\item Fourier transforms?
\end{itemize}

\end{minipage}

\end{frame}

 
\makeatletter{}\begin{frame}
\begin{block}{Gung-Ho: The UM Dynamical Core Project}

\begin{itemize}
\pause
\item Joint project lead by Nigel Wood at the
    \begin{itemize}
    \item UK Met Office
    \item 11 academics funded by Natural Environment Research Council
    \item Science and Technology Facilities Council
    \end{itemize}
\pause
\item Research, design and develop a new dynamical core for:
    \begin{itemize}
    \item operational, global and regional, weather and climate
    \item massively parallel computers
    \end{itemize}
\end{itemize}
\end{block}
\pause
\begin{block}{Requirements:}
\pause
\begin{itemize}
\item Unified approach: 100m -- 150km
\item $10^5$ -- $10^6$ processors
\item Inherent conservation of tracers {\color{purple}\it (and mass?)}
\item Comparable accuracy for wall clock time to current dynamical core
\item Regional modelling (variable grid or limited area)
\item Model top up to 600km
\end{itemize}
\end{block}

\end{frame}

 
\makeatletter{}\begin{frame}
\frametitle{Numerical Requirements (and Mimetic Properties)}
\pause
\begin{itemize}
\item Quasi-uniform tessellation of the sphere (not lat-lon) \pause
\item Approaching 2nd order accuracy \pause
\item Inherent conservation of tracers (and mass) \pause
\item Approaching energy conservation \pause
\item Accurate wave dispersion
    \begin{list}{$\Rightarrow$}
    \item C-grid (staggered grid) rather than A-grid (co-located)\pause
    
        \only<7-8>{\begin{tabular}{cc}
        A-grid & C-grid \\
        \only<7>{\makeatletter{}\begin{picture}(0,0)\includegraphics{../../SS/HilaryNotes/figs/Arakawa/A.pdf}\end{picture}\setlength{\unitlength}{1579sp}\begingroup\makeatletter\ifx\SetFigFont\undefined\gdef\SetFigFont#1#2#3#4#5{  \reset@font\fontsize{#1}{#2pt}  \fontfamily{#3}\fontseries{#4}\fontshape{#5}  \selectfont}\fi\endgroup\begin{picture}(4824,4824)(-11,-3973)
\put(3076,-1936){\makebox(0,0)[b]{\smash{{\SetFigFont{10}{12.0}{\rmdefault}{\mddefault}{\updefault}{\color[rgb]{0,0,0}$h$, $u$, $v$}}}}}
\end{picture} }
        \only<8>{\animategraphics[width=0.3\linewidth]{5}
        {../../../2011/climateSummerSchool/HilaryNotes/2dSWE/Agrid}{0}{9}}
        &
        \only<7>{\makeatletter{}\begin{picture}(0,0)\includegraphics{../../SS/HilaryNotes/figs/Arakawa/C.pdf}\end{picture}\setlength{\unitlength}{1579sp}\begingroup\makeatletter\ifx\SetFigFont\undefined\gdef\SetFigFont#1#2#3#4#5{  \reset@font\fontsize{#1}{#2pt}  \fontfamily{#3}\fontseries{#4}\fontshape{#5}  \selectfont}\fi\endgroup\begin{picture}(4824,4824)(-11,-3973)
\put(2401,-1411){\makebox(0,0)[b]{\smash{{\SetFigFont{10}{12.0}{\rmdefault}{\mddefault}{\updefault}{\color[rgb]{0,0,0}$h$}}}}}
\put(1351,-1861){\makebox(0,0)[b]{\smash{{\SetFigFont{10}{12.0}{\rmdefault}{\mddefault}{\updefault}{\color[rgb]{0,0,0}$u$}}}}}
\put(3301,-1861){\makebox(0,0)[b]{\smash{{\SetFigFont{10}{12.0}{\rmdefault}{\mddefault}{\updefault}{\color[rgb]{0,0,0}$u$}}}}}
\put(2101,-2686){\makebox(0,0)[b]{\smash{{\SetFigFont{10}{12.0}{\rmdefault}{\mddefault}{\updefault}{\color[rgb]{0,0,0}$v$}}}}}
\put(2101,-661){\makebox(0,0)[b]{\smash{{\SetFigFont{10}{12.0}{\rmdefault}{\mddefault}{\updefault}{\color[rgb]{0,0,0}$v$}}}}}
\end{picture} }
        \only<8>{\animategraphics[width=0.3\linewidth]{5}
        {../../../2011/climateSummerSchool/HilaryNotes/2dSWE/Cgrid}{0}{9}}
        \end{tabular}}
        
    \only<8>{\footnotesize\hspace{-1.1cm} (Solution of the linearised shallow-water equations starting from an initial bump)}

    \end{list}\pause\pause
    \item Charney-Phillips staggering in the vertical (pressure and density staggered to avoid spurious hydrostatic balance)\pause
    \item Curl free gradients\pause
    
    \item Exact numerical solutions for geostropically balanced flow $2\Omega\times\mathbf{u}=-g\nabla h$\pause
    \item Accurate implied advection of potential vorticity\pause
    \item Large Courant number in the vertical\pause
    \item<only@14> Control of or lack of spurious computational modes
\end{itemize}


\end{frame}

 
\makeatletter{}\begin{frame}
\frametitle{What Horizontal Grid?}

\newcommand{\figWidth}{0.3\linewidth}
\begin{tabular}{ccc}
\includegraphics[width=\figWidth]
{../../../2011/gungHoNCASJul2011/graphics/shallowWaterTRiSK+WilliSteady+24x48_V2+constant+mesh.pdf}
&
\includegraphics[width=\figWidth]
{../../../2011/gungHoNCASJul2011/graphics/shallowWaterTRiSK+WilliSteady+bucky4+constant+mesh.pdf}
&
\includegraphics[width=\figWidth]
{../../../2011/gungHoNCASJul2011/graphics/shallowWaterTRiSK+WilliSteady+tri4+constant+mesh.pdf}
\\
skipped lat-lon & hexagonal-icosahedral & triangular icosahedral \\
\includegraphics[width=\figWidth]
{../../../2011/gungHoNCASJul2011/graphics/shallowWaterTRiSK+WilliSteady+kite4+constant+mesh.pdf}
&
\includegraphics[width=\figWidth]
{../../../2011/gungHoNCASJul2011/graphics/shallowWaterTRiSK+WilliSteady+cube12_Voronoi+constant+meshBlack.pdf}
&
\includegraphics[width=\figWidth]{../../../2011/gungHoNCASJul2011/figs/yinYang.png}
\\
kites & cubed sphere & Yin-Yang
\end{tabular}

\end{frame}

 
\makeatletter{}\begin{frame}
\frametitle{TRiSK: C-grid on arbitrary grids of the sphere}

Thuburn, Ringler, Skamarock and Klemp, JCP, 2009

\begin{itemize}
\item Method of reconstructing tangential velocity on a face from surrounding normal components on any {\em\color{magenta} orthogonal, 2d} grid
\item Mimetic properties:
    \begin{itemize}
    \item Maintains stationary geostrophic balance\\
        (since divergence on dual = divergence on primal)
    \item Conserves energy
    \item Conserves potential vorticity
    \end{itemize}
\item Low order accurate (0-2)
\end{itemize}

\begin{minipage}{0.4\linewidth}
\includegraphics[width=\linewidth]{graphics/TRiSK.pdf}
\end{minipage}
\begin{minipage}{0.49\linewidth}\centering
TRiSK defines weights, $w_i$ for:
\[v = \sum_i w_i u_i\]
\end{minipage}
\end{frame}

 
\makeatletter{}\begin{frame}
\includegraphics[angle=270, width=1.01\linewidth]{graphics/gridResults2.png}
\end{frame}
 
\makeatletter{}\begin{frame}
\frametitle{Computational Modes on the Hexagonal C-grid}

\begin{minipage}{0.49\linewidth}\centering
Described by Thuburn, 2008
\includegraphics[scale=0.25]{graphics/Thu08Mode.png}
\pause
\end{minipage}
\begin{minipage}{0.49\linewidth}
Vorticty on dual grid:\\
\includegraphics[width=\linewidth]{graphics/shallowWaterSphereMesh+WilliMountain+bucky7+save+dt100_midPointpv_CN+4320000+vorticityHalfZoom.png}
\pause
\end{minipage}
\begin{minipage}{0.49\linewidth}\raggedright\small
Controlled by new monotonic advection scheme to interpolate potential vorticity from vertices to edges: CLUST\\
\includegraphics[width=\linewidth]{graphics/shallowWaterSphereMesh+WilliMountain+bucky7+save+dt100_quadUpBlend_CN_4320000_vorticityHalfZoom.png}
\pause
\end{minipage}
\begin{minipage}{0.49\linewidth}\centering\small
However Model for Prediction Across Scales (MPAS) use APVM (Lax-Wendroff):\\
\includegraphics[width=\linewidth]{graphics/shallowWaterSphereMesh+WilliMountain+bucky7+save+dt100_APVM_CN+4320000+vorticityHalfZoom.png}
\end{minipage}

\end{frame}

 
\makeatletter{}\begin{frame}

\begin{block}{CLUST: \small\normalfont Continuous Linear-Upwind Stabilised Transport {\footnotesize (Weller, 2012)}}

$q_e = b
\underbrace{\biggl(
    q_u + (\mathbf{x}_e-\mathbf{x}_u)\cdot \nabla_u q
\biggr)}_{\text{linear upwind}}
+ (1-b)
\underbrace{
    \half (q_u + q_d)
}_{\text{centred}}
$

where $b = \frac{1}{2}\frac{|u_n|}{|\mathbf{u}_e|}$

\end{block}

\begin{block}{APVM: Anticipated Potential Vorticity Method}

\begin{minipage}{0.53\linewidth}\raggedright
(Sadourny and Basdevant, 1985, Ringler et al, 2010)

\ \\

$q_e = \half(q_u + q_d) - \half\mathbf{u}_e\cdot\nabla_e q~\Delta t$
\end{minipage}
\hfill
\begin{minipage}{0.45\linewidth}
\makeatletter{}\begin{picture}(0,0)\includegraphics{graphics/cells.pdf}\end{picture}\setlength{\unitlength}{1036sp}\begingroup\makeatletter\ifx\SetFigFont\undefined\gdef\SetFigFont#1#2#3#4#5{  \reset@font\fontsize{#1}{#2pt}  \fontfamily{#3}\fontseries{#4}\fontshape{#5}  \selectfont}\fi\endgroup\begin{picture}(5114,4649)(2859,-4328)
\put(5266,-1006){\makebox(0,0)[lb]{\smash{{\SetFigFont{6}{7.2}{\rmdefault}{\mddefault}{\updefault}{\color[rgb]{0,0,0}${\mathbf x}_u$}}}}}
\put(5678,-2317){\makebox(0,0)[lb]{\smash{{\SetFigFont{6}{7.2}{\rmdefault}{\mddefault}{\updefault}{\color[rgb]{0,0,0}${\mathbf u}_e$}}}}}
\put(5049,-2633){\makebox(0,0)[lb]{\smash{{\SetFigFont{6}{7.2}{\rmdefault}{\mddefault}{\updefault}{\color[rgb]{0,0,0}$q_d$}}}}}
\put(5289,-2971){\makebox(0,0)[lb]{\smash{{\SetFigFont{6}{7.2}{\rmdefault}{\mddefault}{\updefault}{\color[rgb]{0,0,0}${\mathbf x}_d$}}}}}
\put(5056,-2258){\makebox(0,0)[lb]{\smash{{\SetFigFont{6}{7.2}{\rmdefault}{\mddefault}{\updefault}{\color[rgb]{0,0,0}$u_n$}}}}}
\put(5476,-1404){\makebox(0,0)[lb]{\smash{{\SetFigFont{6}{7.2}{\rmdefault}{\mddefault}{\updefault}{\color[rgb]{0,0,0}$q_u$}}}}}
\put(5011,-1906){\makebox(0,0)[lb]{\smash{{\SetFigFont{6}{7.2}{\rmdefault}{\mddefault}{\updefault}{\color[rgb]{0,0,0}${\mathbf x}_e$}}}}}
\end{picture} 
\end{minipage}

Same as Lax-Wendroff!
\end{block}

\pause
TRiSK -- Computational modes controlled with CLUST \tick \smiley

\end{frame}

 
\makeatletter{}\begin{frame}
\frametitle{But TRiSK is low-order accurate \frownie:}

\includegraphics[width=\linewidth]{graphics/TRiSKorder.png}

\end{frame}

 
\makeatletter{}\begin{frame}
\frametitle{Finite Element Discretisations}

Mixed Finite Elements -- Different function spaces for pressure and velocity (to avoid pressure modes of co-located methods)

\pause
\begin{itemize}[<+->]
    \item
           \begin{minipage}[t]{0.65\linewidth}\raggedright
           RT0 on hexagons (Thuburn) \\
           (wrong ratio of pressure/velocty dofs $\therefore$ computational modes)
           \end{minipage}\hfill
           \begin{minipage}[c]{0.3\linewidth}
               \includegraphics[scale=0.4]{graphics/RT0hex.pdf}
           \end{minipage}
    \item
          \begin{minipage}[t]{0.65\linewidth}\raggedright
        RT0 on a cubed sphere (Shipton and Cotter)
          \end{minipage}\hfill
          \begin{minipage}[c]{0.3\linewidth}
              \includegraphics[scale=0.4]{graphics/RT0cube.pdf}
          \end{minipage}
    \item
          \begin{minipage}[t]{0.65\linewidth}\raggedright
              RT1 on a cubed sphere (Melvin et al)
          \end{minipage}\hfill
          \begin{minipage}[c]{0.3\linewidth}
      \includegraphics[scale=0.2]{graphics/RT1.png}
          \end{minipage}
    \item BDFM1 (Cotter)
\end{itemize}
\pause
{\em But} the correct pressure/velocity dof ratio is a necessary but {\em not sufficient} to avoid spurious computational modes.
\end{frame}

\begin{frame}
\includegraphics[width=\linewidth]{graphics/ColinGrids.pdf}
\end{frame}

 
\makeatletter{}\begin{frame}
\frametitle{Test Case Results}
\framesubtitle{Solid-body rotation errors after 5 days (shallow water equations)}

\begin{tabular}{cc}
RMS height error ($\ell_2(h)$)
&
Maximum height error ($\ell_\infty(h)$)
\\
\includegraphics[width=0.49\linewidth]{links/TC2+l2h.pdf}
&
\includegraphics[width=0.49\linewidth]{links/TC2+linfh.pdf}
\end{tabular}

Results from Thuburn, Cotter and Weller

\end{frame}

 
\makeatletter{}\begin{frame}
\frametitle{Extension of C-grid to 3D non-orthogonal grids}

\begin{minipage}[c]{0.6\columnwidth}Following Thuburn and Cotter 2012:
\[
U=HV
\]
where
\begin{tabular}{c}
$U=h\ell\mathbf{u}\cdot\mathbf{n}$\tabularnewline
$V=d\mathbf{u}\cdot\mathbf{m}$\tabularnewline
\end{tabular}\end{minipage}\textbf{\hfill{}}\begin{minipage}[c]{0.38\columnwidth}\makeatletter{}\begin{picture}(0,0)\includegraphics{figs/geometry.pdf}\end{picture}\setlength{\unitlength}{1036sp}\begingroup\makeatletter\ifx\SetFigFont\undefined\gdef\SetFigFont#1#2#3#4#5{  \reset@font\fontsize{#1}{#2pt}  \fontfamily{#3}\fontseries{#4}\fontshape{#5}  \selectfont}\fi\endgroup\begin{picture}(6459,4084)(349,-6373)
\put(2956,-4216){\makebox(0,0)[lb]{\smash{{\SetFigFont{6}{7.2}{\rmdefault}{\mddefault}{\updefault}{\color[rgb]{0,0,0}$\ell$}}}}}
\put(3586,-4021){\makebox(0,0)[lb]{\smash{{\SetFigFont{6}{7.2}{\rmdefault}{\mddefault}{\updefault}{\color[rgb]{0,0,0}$d$}}}}}
\put(1231,-4681){\makebox(0,0)[lb]{\smash{{\SetFigFont{6}{7.2}{\rmdefault}{\mddefault}{\updefault}{\color[rgb]{0,0,0}$h$}}}}}
\put(1231,-5461){\makebox(0,0)[lb]{\smash{{\SetFigFont{6}{7.2}{\rmdefault}{\mddefault}{\updefault}{\color[rgb]{0,0,0}$\mathbf{m}$}}}}}
\put(2161,-6061){\makebox(0,0)[lb]{\smash{{\SetFigFont{6}{7.2}{\rmdefault}{\mddefault}{\updefault}{\color[rgb]{0,0,0}$\mathbf{n}$}}}}}
\end{picture} 
\end{minipage}
\pause
\begin{itemize}[<+->]
\item For mimetic properties, $H$ must be symmetric and positive definite. 
\item Additional desirable property:
\begin{itemize}
\item As $\mathbf{n}\cdot\mathbf{m}\rightarrow1$ locally, $H_{ij}\rightarrow\begin{cases}
\frac{h\ell}{d} & \text{when }i=j\\
0 & \text{otherwise }
\end{cases}$
\item Not possible for a symmetric $H$ unless $\mathbf{n}\cdot\mathbf{m}$
is globally uniform.
\item $\therefore$ use an asymmetric $H$
\end{itemize}
\item $H_{ij}\ne0$ for $i\ne j$ when $\mathbf{n}\cdot\mathbf{m}=1$ could
be problematic in the vertical (large stencil for no additional accuracy)
\end{itemize}
\end{frame}

\begin{frame}
\frametitle{$H$ operator}

\textbf{Dubos/Thuburn $H$ (symmetric):}
\[
U_{e}=H(V_{e})=\sum_{e^{\prime}\left(\ne e\right)}\frac{1}{f_{e^{\prime}}}\frac{\left(V_{e}\mathbf{d}_{e^{\prime}}-V_{e^{\prime}}\mathbf{d}_{e}\right)\cdot\mathbf{d}_{e^{\prime}}}{|\mathbf{d}_{e}\times\mathbf{d}_{e^{\prime}}|}
\]
{\small where $\mathbf{d}_{e}=d_{e}\mathbf{m}_{e}$ for edges, $e^{\prime}$
around the two dual cells neighbouring $e$.}


\pause
\textbf{Asymmetric $H$ }with $H_{ij}=0$ for $i\ne j$ when $\mathbf{n}\cdot\mathbf{m}=1$
(in 2D or 3D):
\begin{itemize}[<+->]
\item First reconstruct dual cell velocity:
\[
\mathbf{u}_{v}=T_{v}^{-1}\sum_{e\in EC(v)}\mathbf{m}_{e}V_{e}\ \text{ where }T_{v}=\sum_{e\in EC(v)}d_{e}\mathbf{m}_{e}\mathbf{m}_{e}.
\]
This least squares fit reconstructs a uniform velocity field exactly. 
\item Next interpolate $\mathbf{u}_{v}$ onto faces between dual cells to
give $\mathbf{u_{e}^{\prime}}$\\
and correct to give $V_{e}=d_{e}\mathbf{u}_{e}\cdot\mathbf{m}_{e}$
exactly at the central edge:
\[
\mathbf{\mathbf{u}}_{e}=V_{e}/d_{e}\mathbf{m}_{e}+\mathbf{u}_{e}^{\prime}-\left(\mathbf{u}_{e}^{\prime}\cdot\mathbf{m}_{e}\right)\mathbf{m}_{e}
\]

\item 
Hence: $U_{e}  =h\ell\mathbf{u}_{e}\cdot\mathbf{n}_{e}
 =h\ell/d_{e}\mathbf{m}_{e}\cdot\mathbf{n}_{e}V_{e}+h\ell\mathbf{u}_{e}^{\prime}\cdot\left(\mathbf{n}_{e}-\mathbf{m}_{e}\left(\mathbf{m}_{e}\cdot\mathbf{n}_{e}\right)\right)$.\\
So  no non-orthogonal correction when $\mathbf{n}\cdot\mathbf{m}=1$
as required.
\end{itemize}

\end{frame}
 
\makeatletter{}\begin{frame}
\frametitle{A New Diamond Grid}

\begin{tabular}{cc}
Cubed sphere & Diamond grid\\
\includegraphics[width=0.48\textwidth]
{\HOME/latex/myPapers/nonorthTrisk/graphics/WilliSteady_cube_6x6_eq_constant_orthogonality}
&
\includegraphics[width=0.48\textwidth]
{\HOME/latex/myPapers/nonorthTrisk/graphics/WilliSteady_diamondCube_6x6_eq_constant_orthogonality}
\\
\multicolumn{2}{c}{
\includegraphics[width=0.6\linewidth]
    {\HOME/latex/myPapers/nonorthTrisk/graphics/WilliSteady_HRbucky_legends_orthogonality}
non-orthogonality (degrees)}\\
\multicolumn{2}{c}{
\includegraphics[width=0.6\linewidth]
    {\HOME/latex/myPapers/nonorthTrisk/graphics/WilliSteady_HRbucky_legends_skewness}
skewness}\\
\end{tabular}


\end{frame}
 
\makeatletter{}\begin{frame}
\frametitle{Solid Body Rotation Errors after 5 days -- Asymmetric $H$}

\begin{minipage}{0.32\linewidth}\centering
Hexagonal \\ 2,562 cells\\10,242 dofs\\
\includegraphics[width=1.6\linewidth]
{\HOME/latex/myPapers/nonorthTrisk/graphics/WilliSteady_HRbucky_5_save_dt1800_asymmetricH_CLUSTPV_CLUSTh_432000_hUdiff.png}
\end{minipage}
\only<2-3>{
\begin{minipage}{0.32\linewidth}\centering
Cubed sphere\\
3,456 cells\\
10,368 dofs\\
\includegraphics[width=1.6\textwidth]
{\HOME/latex/myPapers/nonorthTrisk/graphics/WilliSteady_cube_24x24_eq_save_dt1800_asymmetricH_CLUSTPV_CLUSTh_432000_hUdiff}
\end{minipage}}
\only<3>{
\begin{minipage}{0.32\linewidth}\centering
Diamond grid \\
3,468 cells\\
10,404 dofs
\includegraphics[width=1.6\textwidth]
{\HOME/latex/myPapers/nonorthTrisk/graphics/WilliSteady_diamondCube_17x17_eq_save_dt1800_asymmetricH_CLUSTPV_CLUSTh_432000_hUdiff}
\end{minipage}
}
\includegraphics[width=0.8\linewidth]
{\HOME/latex/myPapers/nonorthTrisk/graphics/WilliSteady_HRbucky_legends_hUdiff_hDiff}m


\end{frame}

\begin{frame}
\frametitle{Solid Body Rotation Errors after 5 days -- Symmetric $H$}

\begin{minipage}{0.32\linewidth}\centering
Hexagonal \\ 2,562 cells\\10,242 dofs\\
\includegraphics[width=1.6\linewidth]
{\HOME/latex/myPapers/nonorthTrisk/graphics/WilliSteady_HRbucky_5_save_dt1800_DubosH_CLUSTPV_CLUSTh_432000_hUdiff.png}
\end{minipage}
\begin{minipage}{0.32\linewidth}\centering
Cubed sphere\\
3,456 cells\\
10,368 dofs\\
\includegraphics[width=1.6\textwidth]
{\HOME/latex/myPapers/nonorthTrisk/graphics/WilliSteady_cube_24x24_eq_save_dt1800_DubosH_CLUSTPV_CLUSTh_432000_hUdiff}
\end{minipage}
\begin{minipage}{0.32\linewidth}\centering
Diamond grid \\
3,468 cells\\
10,404 dofs
\includegraphics[width=1.6\textwidth]
{\HOME/latex/myPapers/nonorthTrisk/graphics/WilliSteady_diamondCube_17x17_eq_save_dt1800_DubosH_CLUSTPV_CLUSTh_432000_hUdiff}
\end{minipage}

\includegraphics[width=0.8\linewidth]
{\HOME/latex/myPapers/nonorthTrisk/graphics/WilliSteady_HRbucky_legends_hUdiff_hDiff}m
\end{frame}

\begin{frame}
\frametitle{Flow over a mid-latitude mountain}
\framesubtitle{Diamond Grid, asymmetric $H$}

\animategraphics[width=\textwidth,palindrome]{10}
{\RUNWM/WilliMountain/diamondCube/48x48_eq/save/dt225_asymmetricH_CLUSTPV_CLUSTh_2/animation/lowRes/vorticity}{0}{100}\\
\includegraphics[width=\textwidth]
{\RUNWM/WilliMountain/diamondCube/48x48_eq/save/dt225_asymmetricH_CLUSTPV_CLUSTh_2/legends/vorticity.pdf}

\end{frame}

\begin{frame}
\frametitle{Flow over a mid-latitude mountain -- Asymmetric $H$}
\framesubtitle{Geopotential height errors as a function of resolution}

\begin{tabular}{cc}
RMS error ($\ell_2(h)$) & Max error ($\ell_\infty(h)$) \\
\includegraphics[width=0.48\textwidth]
{\HOME/latex/myPapers/nonorthTrisk/graphics/WilliMountain_comparePlots_l2h_dofs_asymmetricH_CLUSTPV_CLUSTh.pdf}
&
\includegraphics[width=0.48\textwidth]
{\HOME/latex/myPapers/nonorthTrisk/graphics/WilliMountain_comparePlots_lih_dofs_asymmetricH_CLUSTPV_CLUSTh.pdf}
\end{tabular}

\end{frame}


 
\makeatletter{}\begin{frame}

\begin{tabular}{l|l|l}
 & {\bf Finite Volume C-grid} & {\bf Mixed Finite Element} \\
\hline
Cost & All local operations & Invert global mass matrix \\
\hline
Order of accruacy & 0-2 & 2-3 \\
\hline
Accuracy on test & \multicolumn{2}{c}{similar} \\
cases \\
\hline
Accuracy on multi- & loses accuracy? & maintains accuracy? \\
-resolution meshes & requires a slowly \\
 & varying mesh
\end{tabular}

Can we create useful multi-resolution meshes for FV C-grid and maintain accuracy?

\end{frame}

\begin{frame}
\frametitle{Multi-resolution meshes for FV C-grid}

\begin{minipage}[b]{0.48\linewidth}
\centering\small
\ \\ Hexagonal Icosahedron (642 cells)\\
\includegraphics[width=\linewidth]
{links/shallowWaterSphere+WilliMountain+bisect+4+0+mesh.pdf}
\end{minipage}
\begin{minipage}[b]{0.48\linewidth}
\centering\small
Lloyd's algorithm (Ringler et al) (Expensive) (642 cells)\\
\includegraphics[width=\linewidth]
{links/shallowWaterSphere+WilliMountain+Lloyd+4x2+0+mesh.pdf}
\end{minipage}
\pause
\begin{minipage}{0.48\linewidth}
\centering\small
Add points and smooth {\tiny(Sean Cavany, following Walko and Avissar)} (802 cells)\\
\includegraphics[width=\linewidth]
{links/shallowWaterSphere+WilliMountain+Walko+4x2+0+mesh.pdf}
\end{minipage}
\pause
\begin{minipage}{0.48\linewidth}
\centering\small
Solve Parabolic Monge-Ampere equation (Phil Browne) (642 cells) \\
\includegraphics[width=\linewidth]
{links/shallowWaterSphere+WilliMountain+PMA+4x2+0+mesh.pdf}
\end{minipage}
\end{frame}

\begin{frame}
\frametitle{Mesh generation by solving the Parabolic\\
Monge-Ampere Equation on the sphere}

\begin{tabular}{ll}
$\mathbf{x}$ & location of point on uniform mesh\\
$\mathbf{\eta}$ & location of point on multiresolution mesh\\
\end{tabular}\\
Find mesh potential, $P$, such that:
\begin{equation*}
\eta = \mathbf{x} + \nabla P.
\end{equation*}
Work of Chris Budd, Emily Walsh, Phil Browne (Bath)\\
Find $P$ by solving the parabolic Monge-Ampere equation:\\
\begin{equation*}
\frac{\partial P}{\partial t} = \biggl(M(\nabla P) |H(P)|\biggr)^\half
\end{equation*}
where $M$ is the user defined monitor function (contols mesh spacing)\\
and $H$ is the Hessian of $P$

\end{frame}

\begin{frame}
\frametitle{Flow over a mid-latitude mountain}
\framesubtitle{With coarse multiresolution grids, factor of 2 finer around the mountain}

\begin{minipage}{0.48\linewidth}
Uniform grid, 642 cells \\
\includegraphics[width=\linewidth]{links/shallowWaterSphere+WilliMountain+bisect+4+save+dt1800_asymmetricH_CLUSTPV_CLUSTh+1296000+hError200.png}
\end{minipage}
\visible<2->{
\begin{minipage}{0.48\linewidth}
Lloyd's algorithm, 642 cells \\
\includegraphics[width=\linewidth]{links/shallowWaterSphere+WilliMountain+Lloyd+4x2+save+dt1800_asymmetricH_CLUSTPV_CLUSTh+1296000+hError200.png}
\end{minipage}}
\visible<3->{
\begin{minipage}{0.48\linewidth}
Add points, 802 cells \\
\includegraphics[width=\linewidth]{links/shallowWaterSphere+WilliMountain+Walko+4x2+save+dt1800_asymmetricH_CLUSTPV_CLUSTh+1296000+hError200.png}
\end{minipage}}
\visible<4>{
\begin{minipage}{0.48\linewidth}
PMA, 642 cells \\
\includegraphics[width=\linewidth]{links/shallowWaterSphere+WilliMountain+PMA+4x2+save+dt1800_asymmetricH_CLUSTPV_CLUSTh+1296000+hError200.png}
\end{minipage}}

\includegraphics[width=0.7\linewidth]{links/shallowWaterSphere+WilliMountain+legends+hError200_hError.pdf}  \hfill height error (m)
\end{frame}
 
\makeatletter{}\begin{frame}
\frametitle{Conclusions}

\begin{itemize}[<+->]

\item C-grid on a hexagonal-icosahedron is very low accurate in theory (sometimes zeroth order)

\item BUT results of shallow-water test cases are very accurate using C-grid in comparison to more expensive finite element discretisations

\item Computational modes of hexagonal C-grid contolled using CLUST

\item New more accurate non-orthogonal C-grid proposed

\item Diamond grid more orthogonal than cubed sphere

\item Asymmetric non-orthogonal correction more accurate that symmetric correction despite formal loss of energy conservation

\item Sufficiently smooth multi-resolution meshes can be gerated much more cheaply than have previously been done

\item Gradual mesh refinement can lead to improved accuracy

\end{itemize}

\end{frame}

 

\end{document}
