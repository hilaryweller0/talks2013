\chapter{Linear Advection}
\label{chap:advect}

The concentration of a constituent of the atmosphere is denoted $\phi$. If this constituent is moved around by the wind with velocity $\vec{u}$, but there is no diffusion and no sources or sinks of the constituent, then the concentration at time $t$ and at location $(x,y,z)$ is governed by the linear advection equation:
\begin{equation}
\frac{D\phi}{Dt} = \frac{\partial\phi}{\partial t} + \vec{u}\dprod\nabla\phi = 0
\label{eqn:3dAdvect}
\end{equation}
where $\vec{u}=(u,v,w)$ and $\nabla\phi=\bigl(\frac{\partial\phi}{\partial x},\frac{\partial\phi}{\partial y},\frac{\partial\phi}{\partial z}\bigr)$.  Changes in $\phi$ are produced by the component of the wind in the same direction as gradients of $\phi$. In order to understand why the $\vec{u}\dprod\nabla(\phi)$ term leads to changes in $\phi$, consider a region of polluted atmosphere where the pollutant has concentration contours as shown below:

\begin{minipage}{0.5\linewidth} \vspace{1cm}
\includegraphics[width=\linewidth]{figs/contours.pdf}
\end{minipage}
\hfill
\begin{minipage}{0.43\linewidth}
\subsubsubheading{Exercise:}
Draw on the figure the directions of the gradients of $\phi$ and thus mark with a $+$, $-$ or $0$ locations where $\vec{u}\dprod\nabla\phi$ is positive, negative and zero. Thus deduce where $\phi$ is increasing, decreasing or staying the same based on equation \ref{eqn:3dAdvect}. Hence overlay contours of $\phi$ at a later time.
\end{minipage}

\clearpage
In one spatial dimension, $x$, with constant wind, $u$, equation \ref{eqn:3dAdvect} reduces to:
\begin{equation}
\frac{\partial\phi}{\partial t} + \opttext{u\frac{\partial\phi}{\partial x}=0}
\label{eqn:1dAdvect}
\end{equation}
This equation has an analytic solution. If the initial condition is $\phi(x,0)$ then the solution at time $t$ is $\phi(x,t) = \phi(x-ut,0)$. 

\subsubsubheading{Exercise:} Check that this is the analytic solution.\\
(Hint: define $X=x-ut$ and calculate $\frac{\partial X}{\partial x}$ and $\frac{\partial X}{\partial t}$)\\
\optparagraph{
$\frac{\partial X}{\partial x}=1$ and $\frac{\partial X}{\partial t}=-u$\\
$\implies \frac{\partial \phi}{\partial x}=\frac{\partial \phi}{\partial X}\frac{\partial X}{\partial x} = \frac{\partial \phi}{\partial X}$ and 
$\frac{\partial \phi}{\partial t}=\frac{\partial \phi}{\partial X}\frac{\partial X}{\partial t} = -u\frac{\partial \phi}{\partial X}$\\
$\implies \frac{\partial \phi}{\partial t} + u \frac{\partial \phi}{\partial x} = -u\frac{\partial \phi}{\partial X} + u\frac{\partial \phi}{\partial X} = 0$
}\ \\

\begin{minipage}{\linewidth}
\subsubsubheading{Example:} The initial conditions are shown below and the wind speed is 1m/s to the right. Sketch the solution after 3 seconds\\
\includegraphics[scale=0.4]{figs/initialProfile.pdf}
\end{minipage}

\subsubsubheading{Question:} Is the linear advection equation elliptic, parabolic or hyperbolic? \opttext{hyperbolic}

\section{Eulerian Finite Difference Schemes for Linear Advection}

\subsection{Forward in Time, Backward in Space (FTBS)}

To solve $\frac{\partial\phi}{\partial t} + u\frac{\partial\phi}{\partial x}=0$ numerically, divide space into $N$ equal intervals of size $\Delta x$ so that $x_j=j\Delta x,~j=0,1,2,...,N$. Divide time into time steps $\Delta t$ so that $t_n=n\Delta t,~n=0,1,2,...$. Define $\phi_j^n=\phi(x_j,t_n)$. The forward in time, backward in space scheme is:
\begin{equation}
\text{FTBS:}~~~~~~~~~~~~\frac{\phi^{(n+1)}_j - \phi^{(n)}_j}{\Delta t}  +u\frac{\phi^{(n)}_j - \phi^{(n)}_{j-1}}{\Delta x} = 0
\label{eqn:FTBS}
\end{equation}
This can be re-arranged to get $\phi^{(n+1)}_j$ on the LHS and all other terms on the RHS so that we can calculate $\phi$ at the new time step at all locations based on values at previous time steps. Also in this equation, remove $u$, $\Delta t$ and $\Delta x$ by substituting in the Courant number, $c=\frac{u\Delta t}{\Delta x}$:
\begin{equation}
\phi^{(n+1)}_j = \opttext{
\phi^{(n)}_j -c\bigl(\phi^{(n)}_j - \phi^{(n)}_{j-1}\bigr)
}
\end{equation}

%\vskip 0.54cm
\begin{minipage}{0.48\linewidth}
Advection of an initial profile once around the domain (cyclic boundaries, $c=0.2$, 100 space intervals, 500 time steps).\\
\subsubsubheading{Question} Why is it so inaccurate?\\
\opttext{First-order accurate in space and time.}
\subsubsubheading{Question} Why is it so diffusive?\\

\end{minipage}\hfill
\begin{minipage}{0.5\linewidth}
\animategraphics[width=\linewidth]{5}
{/home/hilary/latex/teaching/MTMW12/2012Notes/notes/anims/FTBS/FTBS_}{0}{49}
\end{minipage}
\optparagraph{The first term omitted from the Taylor series for the  backward in space approximation for $\partial\phi/\partial x$ looks like a diffusion term.}

\clearpage
\subsection{Centred in Time, Centred in Space (CTCS)}

\begin{equation}
\text{CTCS:}~~~~~~~~~~~~\frac{\phi^{(n+1)}_j - \phi^{(n-1)}_j}{2\Delta t}  +u\frac{\phi^{(n)}_{j+1} - \phi^{(n)}_{j-1}}{2\Delta x} = 0
\label{eqn:CTCS}
\end{equation}
\subsubsubheading{Exercise:} Re-arrange to get $\phi^{(n+1)}_j$ on the LHS and all other terms on the RHS. Also remove $u$, $\Delta t$ and $\Delta x$ by substituting in the Courant number, $c=\frac{u\Delta t}{\Delta x}$:
\begin{equation}
\phi^{(n+1)}_j = \opttext{
\phi_j^{(n-1)} - c\bigl(\phi^{(n)}_{j+1} - \phi^{(n)}_{j-1}\bigr)
}
\end{equation}
This is a three-time-level formula (it involves values of $\phi$ at times $t_{n-1}$, $t_n$ and $t_{n+1}$. To start the simulation, values of $\phi$ are needed at times $t_0$ and $t_1$. However, only $\phi(x,t_0)$ is available. So another scheme (such as FTCS) must be used to obtain $\phi(x,t_1)$:
\begin{equation}
\text{FTCS:~~~}\opttext{
\phi^{(n+1)}_j = \phi_j^{(n)} - \frac{c}{2}\bigl(\phi^{(n)}_{j+1} - \phi^{(n)}_{j-1}\bigr)
}
\label{eqn:FTCS}
\end{equation}

\begin{itemize}
\item The two-point, centred difference in space gradient is 2nd order accurate.
\item The centred in time or leap-frog scheme is second-order accurate.
\item So CTCS is also second-order accurate.
\item But is it stable?
\item Are there any other undesirable numerical properties?
\end{itemize}

\clearpage
\subsubheading{CTCS or FTBS?}

Advection of an initial profile once around the domain (cyclic boundaries, c = 0.2, 100 space intervals, 500 time steps).

\animategraphics[width=\linewidth]{5}
{/home/hilary/latex/teaching/MTMW12/2012Notes/notes/anims/FTBS_CTCS/FTBS_CTCS_}{0}{49}

\begin{itemize}
\item CTCS appears to be stable
\item CTCS is not diffusive
\item CTCS suffers from {\it dispersion errors} -- short wavelength waves propagate too slowly. This error is caused by the centred in space discretisation
\end{itemize}

\clearpage
\subsection{The Computational Mode of CTCS}

CTCS uses three time levels to model an equation which is first-order in time (you should only need two time levels to represent $\partial\phi/\partial t$). CTCS can therefore produce spurious solutions which oscillate in time and propagate in the wrong direction. This is called the computational mode. This can be understood by considering the domain of dependence of CTCS.

The solution $\phi^{(n+1)}_{j}$ depends on $\phi^{(n)}_{j\pm 1}$ and $\phi^{(n-1)}_{j}$ but not on $\phi^{(n)}_{j}$. Draw the computational and physical domain of dependence on the graph:

\begin{tabular}{lcr}
CTCS && $\phi^{(n+1)}_j = 
\phi_j^{(n-1)} - c\bigl(\phi^{(n)}_{j+1} - \phi^{(n)}_{j-1}\bigr)$
\end{tabular}

\includegraphics[scale=0.3]{figs/domainDependence.pdf}

Except at the initial time, the solution is found on two sets of points that are not coupled. The solutions on the two sets of points can do two completely different things. They will never see each other.

