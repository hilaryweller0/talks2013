\subsection{Practical: Diffusion}

\begin{enumerate}
\item Write a code to solve the diffusion equation using FTCS and the following parameters

\begin{tabular}{lll}
$x_{\min}$ & 0.0 & limits of the geometry\\
$x_{\max}$ & 1.0 & \\
$n_x$ & 100 & number of spatial points \\
$n_t$ & 50 & number of time-steps \\
$K$ & 1.0 & diffusion coefficient \\
$\Delta t$ & $2\times 10^5$ & Time step \\
\end{tabular}

Starting from the initial conditions:
\begin{equation*}
\phi_0(x) = \begin{cases}
1 & 0.25<x<0.75 \\
0 & \text{ otherwise}
\end{cases}
\end{equation*}
And subject to boundary conditions $\phi(0)=\phi(1)=0$ for all time. Experiment to see what happens when the time-step is increased above the stability limit.

\item BTCS is solved by formulating an $N\times N$ matrix to solve where $N$ is the number of grid-points. Write out BTCS as an $N\times N$ matrix equation for the vector of all $\phi_j^{(n+1)}$s in terms of all of the $\phi_j^{(n)}$s. 

\item Matrices can be initialised in python by using:
\begin{lstlisting}
M = pylab.zeros([N,N])
\end{lstlisting}
and matrix equations can be solved by importing the following libraries:
\begin{lstlisting}
import scipy as Sci
import scipy.linalg
\end{lstlisting}
Then the solution of $Mx=y$ is
\begin{lstlisting}
x = scipy.linalg.solve(M,y)
\end{lstlisting}
Use this to implement BTCS for solving the diffusion equation. How do the answers compare with FTCS when the time-step is below the stability limit. What happens when you increase the time-step?

\end{enumerate}

\subsection{Lessons from the practical}


