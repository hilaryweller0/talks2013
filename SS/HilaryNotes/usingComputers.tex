\clearpage
.
\clearpage
{\bf\large Using the Computers for the Practicals}

\begin{enumerate}
\item For each practical, make a new directory in your home directory, or a sub-directory in your home directory, eg:

{\tt mkdir prac1\_linearAdvection}

{\tt cd prac1\_linearAdvection}

\item Copy the appropriate files from\\
{\tt /usr/local/atmos/CMSS2013\_numerics/pythonExamples}. For example, to copy the whole {\tt advection} directory:

{\tt cp -r /usr/local/atmos/CMSS2013\_numerics/pythonExamples/advection $\sim$/prac1\_linearAdvection}

($\sim$ is a shortcut for your home directory)

\item Go to this directory, make a copy of the file that you want to edit and edit it using {\tt gedit}:

{\tt cd advection}

{\tt ls}

{\tt cp linearAdvect.py linearAdvect\_HW.py}    \hfill (HW are my initials)

{\tt gedit linearAdvect\_HW.py \&}

(The {\tt \&} makes the command run in the background)

\item {\bf Very important: tabs versus spaces.} If you mix tabs and spaces in python, it will behave in unexpected ways. The python code that I am giving you has spaces not tabs. So you must setup gedit to put in 4 spaces instead of a tab. In {\tt gedit} go to {\tt Edit -> Preferences -> Editor} and choose {\tt Insert spaces instead of tabs} and set the tab width to 4. You can also enable automatic indentation.

\item The python scripts that I am giving you are heavily commented so that you can pick up the syntax from the script. So have a good look at them before making any changes. 

\item Once you have made and saved your changes you can start python and run your script:

{\tt python}

and in python:

{\tt execfile("linearAdvect\_HW.py")}

\item In python, after a graph has been opened in a new window, you must close that window before the script will continue

\item To leave python, type {\tt exit()}

\item Some essential unix commands:

\begin{tabular}{ll}
{\tt mkdir dirName} & Make directory named ``dirName'' \\
{\tt cd dirName} & Change directory to directory named ``dirName''\\
{\tt cp file1 file2} & Copy file named ``file1'' to location ``file2''\\
{\tt cp -r dir1 dir2} & Copy the entire directory ``dir1'' to location ``dir2''\\
{\tt ls} & List contents of the current directory \\
{\tt pwd} & Print working directory \\
{\tt mv file1 file2} & Move file (or directory) ``file1'' to file or directory named ``file2'' \\
{\tt rm file1} & Remove (delete) ``file1'' \\
{\tt rm -r dir1} & Remove (delete) directory structure ``dir1'' \\
{\tt man command} & Print the manual page for ``command''\\
{\tt more file1} & Write out the contents of ``file1'' one page at a time\\
& (get another page by pressing space) \\
\end{tabular}

\item You can use {\tt okular} to view pdf files. Eg:

{\tt okular Weller\_ClimSS\_lec.pdf \&}

\end{enumerate}

